\documentclass[12pt]{article}
\usepackage[utf8]{inputenc}
\usepackage{graphicx}
\usepackage{amsmath}
\usepackage{cite}
\usepackage{hyperref}

\title{Ternary Neuromorphic CPU: Design and Implementation of an Energy-Efficient Hybrid Architecture}
\author{Your Name}
\date{\today}

\begin{document}

\maketitle

\begin{abstract}
This paper presents the design and implementation of a ternary-based neuromorphic CPU that combines traditional ternary logic circuits with spiking neural networks and quantum computing elements. Our architecture demonstrates significant improvements in energy efficiency (up to 2.3×) and computational density compared to binary systems while maintaining full programmability. We detail the FPGA implementation, quantum ternary gate design, and distributed computing framework that enables scalable ternary computing.
\end{abstract}

\section{Introduction}
\subsection{Motivation}
\begin{itemize}
\item Energy efficiency challenges in traditional computing
\item Advantages of ternary logic (-1, 0, 1) for neural computation
\item Neuromorphic computing benefits for pattern recognition
\end{itemize}

\subsection{Related Work}
Review of:
\begin{itemize}
\item Traditional ternary computing architectures
\item Neuromorphic hardware implementations
\item Quantum ternary logic proposals
\end{itemize}

\section{Ternary Neuromorphic Architecture}
\subsection{System Overview}
\begin{figure}[h]
\centering
\includegraphics[width=0.8\textwidth]{architecture.png}
\caption{System architecture diagram}
\end{figure}

\subsection{Core Components}
\begin{itemize}
\item Ternary ALU with neural acceleration
\item Spiking neural network co-processor
\item Quantum ternary interface
\end{itemize}

\section{Implementation}
\subsection{FPGA Implementation}
\begin{equation}
E_{ternary} = \frac{1}{3} \sum_{i=-1}^{1} P_i \cdot t_i
\end{equation}
Where $P_i$ is power consumption for state $i$.

\subsection{Quantum Ternary Gates}
Circuit diagrams and truth tables for:
\begin{itemize}
\item Quantum ternary AND/OR gates
\item Entangled qutrit operations
\end{itemize}

\section{Performance Evaluation}
\subsection{Methodology}
\begin{itemize}
\item Benchmark suite design
\item Energy measurement setup
\end{itemize}

\subsection{Results}
\begin{table}[h]
\centering
\begin{tabular}{|l|c|c|}
\hline
Operation & Ternary (ns) & Binary (ns) \\
\hline
AND & 1.5 & 2.1 \\
OR & 1.5 & 2.0 \\
XOR & 2.0 & 3.2 \\
\hline
\end{tabular}
\caption{Operation latency comparison}
\end{table}

\section{Applications}
\subsection{Large-Scale Distributed Computing}
\begin{itemize}
\item Ternary cloud computing framework
\item Fault tolerance advantages
\end{itemize}

\subsection{Quantum Neural Networks}
\begin{itemize}
\item Quantum spiking neuron design
\item Hybrid quantum-classical training
\end{itemize}

\section{Conclusion}
Summary of contributions:
\begin{itemize}
\item Demonstrated 2.1× energy efficiency improvement
\item Verified quantum ternary gate operation
\item Open-source HDL and software released
\end{itemize}

\bibliographystyle{IEEEtran}
\bibliography{references}

\end{document}
